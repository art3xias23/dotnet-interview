\chapter*{HTTP Headers} % or \section*{} depending on your document class
\hypertarget{page1}{} % Target for hyperlink in the main document

\section*{GET} % Section for Header information
\begin{itemize}
    \item Used to retrieve data from a server.
    \item Safe operation as it does not change the state of any resource on the server.
\end{itemize}

\vspace{12pt} % Add vertical space between sections (optional)

\section*{POST} % Section for Bay information
\begin{itemize}
    \item Data sent to the server in the following form: 
    \begin{itemize}
        \item Input fields from online forms.
        \item XML or JSON.
        \item Text data from query parameters.
    \end{itemize}
    \item Not a safe operation as it could change the state of the server.
    \item Not idempotent.
\end{itemize}

\section*{HEAD}
\begin{itemize}
    \item Returns the headers of the resource, but not the actual resource.
\end{itemize}

\section*{PUT}
\begin{itemize}
    \item Always includes a payload containing a new resource definition which is meant to replace the one on the server.
    \item It sends a full resource, not like PATCH which sends a partial resource.
    \item A PUT operation always uses an exact URL to target a resource. If that resource does not exist a new one is created.
    \item Unsafe but Idempotent.
\end{itemize}

\section*{PATCH}
\begin{itemize}
    \item Like PUT it's meant to update a resource. However, sometimes it's more efficient to send a small payload than a complete resource.
\end{itemize}

\section*{DELETE}
\begin{itemize}
    \item The resource the url points to is removed after execution.
    \item Unsafe and Idempotent
\end{itemize}

\section*{TRACE}
\begin{itemize}
    \item Used for diagnostics.
    \item Echoes back from the server the headers that the client sent.
\end{itemize}

\section*{OPTIONS}
\begin{itemize}
    \item A server does not need to support every HTTP method for every resource it manages. Sometimes it might only support GET or POST.
    \item HTTP OPTIONS returns a list of which HTTP methods are supported and allowed.
\end{itemize}