\chapter{Dictionary}


\section{Handling Mechanism}
	\begin{itemize}
		\item Uses hash tables internally.
		\item When we add a key-value pair to a dictionary, the key is hashed which produces as \underline{index}.
			\begin{lstlisting}
var hashInt = "Key1".GetHashCode();
// returns 170634988 in .NET 8

var hashInt = 3.GetHashCode();
// returns 3 as integers return themselves
			\end{lstlisting}
		\item The system also assigns a number of buckets (places in memory) which will hold the key-value pairs.
		\item Items that generate different hash codes will be placed in different buckets.
		\item The dictionary decides which bucket to place the key-value entry in by using the modulus operation:
			\begin{lstlisting}
int index = hashCode % numberOfBuckets;
			\end{lstlisting}
		\item The structure of the entry stored in a bucket consists of:
		\begin{itemize}
			\item \textbf{Key}: The original key used to store and retrieve the value.
			\item \textbf{Value}: The value associated with the key.
			\item \textbf{Next Entry}: The reference to the next item in the case of a \underline{collision}.
		\end{itemize}
		\item Items that generate the same hash code will be placed in the same bucket. These are known as \underline{collisions}.
	\end{itemize}

\section{Collision Handling}
	\begin{itemize}
		\item Collisions occur when two keys produce the same hashcode. Below is an example simplified formula
		\begin{lstlisting}
	 static int GetStableHashCode(string str)
    {
        {
            int hash = 23;
						// Random prime number is used
            foreach (var c in str)
            {
                hash = hash * 31 + c;
								// again we use 31 as it offers good balance between performance and distribution(less chance of collisions)
            }
            return hash;
        }
    }
		\end{lstlisting}
		\item Collisions slow down dictionary retrieval, as a linked list is used to store items with the same hash code in the same bucket.
		\item



